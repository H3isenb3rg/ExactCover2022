\section{Algoritmo}
\subsection{Principi}
L’algoritmo risolvente proposto (in due versioni) sfrutta l’ordine degli insiemi entro la collezione N (detto \underline{ordine lessicografico}).\\
Esso produce incrementalmente la matrice di compatibilità e intanto analizza via via gli aggregati di insiemi della collezione N la compatibilità reciproca dei quali sia già stata appurata.
\begin{itemize}
    \item Nessun aggregato contiene come elemento un insieme vuoto
    \item Nessun aggregato contiene come elemento un insieme coincidente con M
\end{itemize}
L’operato dell’algoritmo può essere descritto in termini di \underline{\textbf{esplorazione di alberi}}, dove un nodo d’albero rappresenta un aggregato di uno o più insiemi della collezione.
\begin{itemize}
    \item Ciascun albero ha per radice un insieme distinto della collezione e contiene tutti e soli gli aggregati costituiti da tale insieme e da insiemi che lo precedono secondo l’ordine lessicografico degli insiemi.
    \item A ogni nodo d’albero è associato l’insieme degli identificatori degli insiemi che appartengono all’aggregato a cui il nodo si riferisce
    \item A ciascuna radice è associato un insieme (di identificatori) singoletto $\{i\}$, $1 \leq i \leq |N|$, dove $A[i]$ non è l’insieme vuoto né coincide con M: l’albero avente tale radice si dice «\underline{radicato in i}»
    \item Gli alberi vengono visitati per valori crescenti degli identificatori delle loro radici (ordine lessicografico)
\end{itemize}

\subsection{Visita degli Aggregati}
\paragraph{Due insiemi.}
Durante la visita dell’albero radicato in $i$ vengono inizializzate le caselle della matrice di compatibilità della colonna relativa a $i$ (prima la casella $B[1,i]$, poi $B[2,i]$ ecc. fino ad assegnare un valore alla casella $B[i-1,i]$): questo implica \textbf{la visita di tutti i nodi corrispondenti ad aggregati di cardinalità 2}
\paragraph{Tre o più insiemi.}
Ciascun nodo corrispondente a un aggregato di cardinalità $>2$ viene invece visitato \textbf{solo se gli insiemi che lo costituiscono sono compatibili a due a due}, secondo quanto riportato nella matrice B
\paragraph{Sotto-colonne}
Nello pseudocodice, con $B[1...i, j]$ si indica la sotto collezione di N costituita dagli insiemi compatibili con l’insieme $A[j]$ i cui indici vanno da 1 a $i (j>i)$

\subsection{Ordine di Visita e Analisi}
\begin{figure}[H]
  \centering
  \includesvg[inkscapelatex=false, scale=0.60]{figures/Tree.svg}
  \caption{Esempio Albero dell'Algoritmo}
\end{figure}
Vicino ai nodi è riportato l’ordine di visita dell’albero radicato \textbf{nell’insieme identificato da 5}, assumendo che gli insiemi non vuoti né coincidenti con M che lo precedono nella collezione N siano quelli aventi gli identificatori da 1 a 4 e che l’ordine entro la collezione rispecchi l’ordine dei valori interi.
\begin{itemize}
    \item L’ordine di visita illustrato garantisce che le caselle $B[1,5]$, $B[2,5]$, $B[3,5]$ e $B[4,5]$ vengano inizializzate nell’ordine indicato
    \item Dopo che sono state inizializzate le caselle $B[1,5]$ e $B[2,5]$ è possibile analizzare l’aggregato 521 perché le compatibilità degli aggregati 51 e 52 sono note, così come quella dell’aggregato 21, poiché l’albero radicato in 2 è stato visitato prima di quello radicato in 5
    \item Dopo che sono state inizializzate le caselle $B[1,5]$, $B[2,5]$ e $B[3,5]$ è possibile analizzare gli aggregati 531 e 532 perché le compatibilità degli aggregati 51, 52 e 53 sono note, così come quelle degli aggregati 31 e 32
\end{itemize}
\begin{figure}[H]
  \centering
  \includesvg[inkscapelatex=false, scale=0.60]{figures/Tree2.svg}
  \caption{Esempio Albero dell'Algoritmo con nodi scartati}
\end{figure}
\begin{itemize}
    \item Se la visita di un nodo del primo livello di un albero (cioè l’aggregato $ij$ , figlio della radice $i$) evidenzia che gli insiemi $A[i]$ e $A[j]$ hanno \textbf{un'intersezione non vuota} oppure che \textbf{la loro unione coincide con M}, i discendenti del nodo $ij$ non vengono esplorati così come non vengono esplorati gli aggregati che contengono sia $i$ sia $j$.
    \begin{itemize}
        \item Per Esempio se l’insieme 5 ha una intersezione non vuota con l’insieme 3 oppure se l’unione degli insiemi 5 e 3 coincide con M, i nodi discendenti del nodo 53 non vengono visitati così come non vengono visitati i nodi appartenenti all’intero sottoalbero radicato in 543, come evidenziato dal nuovo contorno di tali nodi
    \end{itemize}
    \item Se la visita di un nodo appartenente a un livello successivo al primo evidenzia che l’aggregato rappresentato da tale nodo \textbf{è una partizione}, i discendenti di tale nodo non vengono esplorati.
    \begin{itemize}
        \item Per Esempio se l’aggregato 542 è una partizione, i suoi nodi discendenti (in questo caso uno solo) non vengono visitati
    \end{itemize}
\end{itemize}

\newpage
\subsection{Algoritmo Base - Pseudocodice}
\IncMargin{1em}
\begin{algorithm}
    \DontPrintSemicolon
    \SetKwArray{A}{A}\SetKwData{i}{i}\SetKwArray{Rows}{rows}
    \SetKwData{SetM}{M}\SetKwData{Cov}{COV}\SetKwArray{B}{B}
    \SetKwData{j}{j}\SetKwData{U}{U}\SetKwData{I}{I}\SetKwData{Itemp}{Itemp}
    \SetKwData{Utemp}{Utemp}\SetKwData{Inter}{Inter}\SetKwData{VarK}{k}
    \SetKwData{InterTemp}{Intertemp}
    \SetKwFunction{Esplora}{Esplora}
    \BlankLine
    \SetKwProg{Fn}{procedure}{}{}
    \Fn{EC(\A)}{
        \For{\i$\leftarrow1$ \KwTo \Rows{\A}}{
            \lIf(\tcc*[f]{\textbf{break} termina iterazione $i$-esima}){\A{\i}$==\varnothing$}{
                \textbf{break}
            }
            \lIf(\tcc*[f]{\Cov variabile globale}){\A{\i}$==$\SetM}{
                Put $\{i\}$ in \Cov,
                \textbf{break}
            }
            In \B aggiungere la colonna relativa ad \i\;
            \For{\j$\leftarrow1$ \KwTo \i$-1$}{
                \eIf{\A{\j}$\cap$\A{\i}$\neq\varnothing$}{
                    \B{\j,\i}$\leftarrow0$
                }{
                    \I$\leftarrow\{$\i,\j$\}$, \U$\leftarrow$\A{\i}$\cup$\A{\j}\;
                    \eIf{\U$==$\SetM}{
                        inserire \I in \Cov, \B{\j,\i}$\leftarrow0$
                    }{
                        \B{\j,\i}$\leftarrow1$, \Inter $\leftarrow$ \B{$1...$\j$-1$, \i}$\cap$\B{$1...$\j$-1$, \j}\;
                        \lIf{\Inter$\neq\varnothing$}{
                            \Esplora{\I, \U, \Inter}
                        }
                    }
                }
            }
        }
    }
    \Fn{Esplora(\I, \U, \Inter)}{
        \ForAll(\tcc*[f]{Ordine lessicografico del valore di \VarK}){\VarK$\in $\Inter }{
            \Itemp$\leftarrow$\I$\cup\{$\VarK$\}$, 
            \Utemp$\leftarrow$\U$\cup$\A{\VarK}\;
            \eIf{\Utemp$==$\SetM}{
                inserire \Itemp in \Cov
            }{
                \InterTemp $\leftarrow$ \Inter $\cap$ \B{$1...$\VarK$-1$,\VarK}\;
                \lIf{\InterTemp$\neq\varnothing$}{\Esplora{\I, \U, \Inter}}
            }
        }
    }
    \caption{Algoritmo Base}\label{base_alg}
\end{algorithm}\DecMargin{1em}

\newpage
\subsection{Algoritmo Plus - Pseudocodice}
\IncMargin{1em}
\begin{algorithm}
    \DontPrintSemicolon
    \SetKwArray{A}{A}\SetKwArray{Rows}{rows}\SetKwArray{B}{B}
    \SetKwArray{Card}{card}
    \SetKwData{SetM}{M}\SetKwData{Cov}{COV}\SetKwData{i}{i}
    \SetKwData{j}{j}\SetKwData{CardU}{cardU}\SetKwData{I}{I}\SetKwData{Itemp}{Itemp}
    \SetKwData{Utemp}{Utemp}\SetKwData{Inter}{Inter}\SetKwData{VarK}{k}
    \SetKwData{CardTemp}{cardtemp}
    \SetKwData{InterTemp}{Intertemp}
    \SetKwFunction{Esplora}{Esplora$^+$}
    \BlankLine
    \SetKwProg{Fn}{procedure}{}{}
    \Fn{EC$^+$(\A)}{
        \For{\i$\leftarrow1$ \KwTo \Rows{\A}}{
            \lIf(\tcc*[f]{\textbf{break} termina iterazione $i$-esima}){\A{\i}$==\varnothing$}{
                \textbf{break}
            }
            \lIf(\tcc*[f]{\Cov variabile globale}){\A{\i}$==$\SetM}{
                Put $\{i\}$ in \Cov,
                \textbf{break}
            }
            \Card{\i}$\leftarrow|$\A{\i}$|$\;
            In \B aggiungere la colonna relativa ad \i\;
            \For{\j$\leftarrow1$ \KwTo \i$-1$}{
                \eIf{\A{\j}$\cap$\A{\i}$\neq\varnothing$}{
                    \B{\j,\i}$\leftarrow0$
                }{
                    \I$\leftarrow\{$\i,\j$\}$, 
                    \CardU$\leftarrow$\Card{\i}$+$\Card{\j}\;
                    \eIf{\CardU$==|$\SetM$|$}{
                        inserire \I in \Cov, \B{\j,\i}$\leftarrow0$
                    }{
                        \B{\j,\i}$\leftarrow1$, \Inter $\leftarrow$ \B{$1...$\j$-1$, \i}$\cap$\B{$1...$\j$-1$, \j}\;
                        \lIf{\Inter$\neq\varnothing$}{
                            \Esplora{\I, \CardU, \Inter}
                        }
                    }
                }
            }
        }
    }
    \Fn{Esplora$^+$(\I, \CardU, \Inter)}{
        \ForAll(\tcc*[f]{Ordine lessicografico del valore di \VarK}){\VarK$\in $\Inter }{
            \Itemp$\leftarrow$\I$\cup\{$\VarK$\}$, 
            \CardTemp$\leftarrow$\CardU$+|$\A{\VarK}$|$\;
            \eIf{\CardTemp$==|$\SetM$|$}{
                inserire \Itemp in \Cov
            }{
                \InterTemp $\leftarrow$ \Inter $\cap$ \B{$1...$\VarK$-1$,\VarK}\;
                \lIf{\InterTemp$\neq\varnothing$}{\Esplora{\I, \CardTemp, \Inter}}
            }
        }
    }
    \caption{Algoritmo Plus}\label{plus_alg}
\end{algorithm}\DecMargin{1em}