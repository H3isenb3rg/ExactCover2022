\chapter{Descrizione del progetto}

\section{Il Problema}

\subsection{Input}
Una \textbf{collezione finita N di insiemi finiti} (distinti) dove gli elementi di ogni insieme appartengono al dominio M (si assume che M sia l’unione di tutti gli insiemi della collezione N)

\subsection{Output}
Tutte le \textbf{partizioni} (o \textbf{coperture esatte}) di M dove ciascuna parte è un insieme della collezione N

\textbf{Partizione di M} (vincolata da N):
\begin{itemize}
    \item (sotto)insieme della collezione N costituito da \underline{insiemi tutti reciprocamente disgiunti} e tali che \underline{la loro unione è M}
\end{itemize}

\subsection{Esempio}
\newcommand\xxx{\par\hangindent1em\makebox[1em][l]{}}
\begin{tabular}{p{0.15\textwidth}p{0.5\textwidth}lp{2in}}\toprule
    N & 
        \xxx \{b3\}
        \xxx \{a1, a2, b4\}
        \xxx \{a2, b3, b4, a5\}  
        \xxx \{a5\}  
        \xxx \{a1, a2, b3\}  
        \xxx \{b4, a5\} \\
    \addlinespace
    M & 
        \xxx \{a1,a2,b3,b4,a5\}\\
    \addlinespace
    Partizioni & 
        \xxx \{\{b3\}, \{a1,a2,b4\}, \{a5\}\}  
        \xxx \{\{a1,a2,b3\}, \{b4,a5\}\}\\
    \\\bottomrule
\end{tabular}

\section{Componenti}

\subsection{Matrice d'Ingresso}
Ogni elemento di M è univocamente identificato da un indice intero appartenente all’intervallo [1 .. |M|]
\begin{align*}
 & a1 \rightarrow 1 \\
 & a2 \rightarrow 2 \\
 & b3 \rightarrow 3 \\
 & b4 \rightarrow 4 \\
 & a5 \rightarrow 5 \\
\end{align*}
Analogamente ogni elemento di N è univocamente identificato da un indice intero appartenente all’intervallo [1 .. |N|]
\begin{align*}
     & \{b3\} \rightarrow 1 \\
     & \{a1,a2,b4\} \rightarrow 2 \\ 
     & \{a2,b3,b4,a5\} \rightarrow 3 \\ 
     & \{a5\}  \rightarrow 4 \\
     & \{a1,a2,b3\} \rightarrow 5 \\
     & \{b4,a5\} \rightarrow 6 
\end{align*}
I dati d’ingresso del problema possono essere rappresentati come una matrice $A_{|N|,|M|}$, dove il valore del componente ai,j della matrice è 1 se l’elemento (di M) di indice j appartiene all’insieme (in N) di indice i, 0 altrimenti.\\
\underline{Precisazione}: un insieme vuoto può comparire in ogni partizione, tuttavia è preferibile ometterlo. Pertanto, se un insieme delle collezione N fosse accidentalmente vuoto, esso non dovrà comparire in nessuna soluzione del problema Exact Cover.
\[
\begin{blockarray}{cccccc}
     & a1 & a2 & b3 & b4 & a5 \\
     & 1 & 2 & 3 & 4 & 5 \\
    \begin{block}{c[ccccc]}
        1 & 0 & 0 & 1 & 0 & 0\bigstrut[t] \\
        2 & 1 & 1 & 0 & 1 & 0 \\
        3 & 0 & 1 & 1 & 1 & 1 \\
        4 & 0 & 0 & 0 & 0 & 1 \\
        5 & 1 & 1 & 1 & 0 & 0 \\
        6 & 0 & 0 & 0 & 1 & 1\bigstrut[b]\\
    \end{block}
\end{blockarray}\vspace*{-1.25\baselineskip}
\]
\\
D’ora innanzi si farà riferimento all’insieme i-mo $(1 \leq i \leq |N|)$ della collezione N come A[i]

\subsection{COV}
Insieme (denominato COV) di tutte le partizioni trovate, dove ciascuna partizione è rappresentata da un insieme di identificatori, uno per ciascun insieme appartenente a N che fa parte della partizione\\
\begin{tabular}{p{0.15\textwidth}p{0.5\textwidth}lp{2in}}
    \toprule
    N & 
        \xxx 1 $\rightarrow$ \{b3\}
        \xxx 2 $\rightarrow$ \{a1, a2, b4\}
        \xxx 3 $\rightarrow$ \{a2, b3, b4, a5\}  
        \xxx 4 $\rightarrow$ \{a5\}  
        \xxx 5 $\rightarrow$ \{a1, a2, b3\}  
        \xxx 6 $\rightarrow$ \{b4, a5\} \\
    \addlinespace
    Partizioni & 
        \xxx \{\{b3\}, \{a1,a2,b4\}, \{a5\}\}  
        \xxx \{\{a1,a2,b3\}, \{b4,a5\}\}\\
    \bottomrule
    \textbf{COV} &
        \xxx \{1, 2, 4\}  
        \xxx \{5, 6\}\\
    \bottomrule
\end{tabular}

\subsection{Matrice di Compatibilità}
L’approccio proposto fa ampio uso del concetto di matrice di compatibilità, che è relativa agli insiemi della collezione distinti da $\varnothing$ e $M$

\paragraph{Matrice di Compatibilità}
Si tratta di una matrice simmetrica, indicata con $B$, in cui $b_{i,j}$ assume:
\begin{itemize}
    \item Il valore 1 se:
    \begin{itemize}
        \item $i\neq j$
        \item $A[i] \cap A[j] = \varnothing$
        \item $A[i] \cup A[j] \neq M$
    \end{itemize}
    \item 0 altrimenti
\end{itemize}

Della matrice B basta riempire le celle $b_{i,j}$ in cui $j>i$ (La matrice è simmetrica).\\
$A[i]$ e $A[j]$ (con $j>i$) si dicono insiemi compatibili se $b_{i,j}=1$ (nel senso che ad essi si possono aggregare ulteriori insiemi al fine di formare delle partizioni)

\begin{center}
    \begin{tabular}{p{0.35\textwidth}p{0.3\textwidth}lp{2in}}
        \begin{center}Matrice di Ingresso A\end{center} & \begin{center}Matrice di Compatibilità B\end{center}\\
        \[
        \begin{blockarray}{cccccc}
             & a1 & a2 & b3 & b4 & a5 \\
             & 1 & 2 & 3 & 4 & 5 \\
            \begin{block}{c[ccccc]}
                1 & 0 & 0 & 1 & 0 & 0\bigstrut[t] \\
                2 & 1 & 1 & 0 & 1 & 0 \\
                3 & 0 & 1 & 1 & 1 & 1 \\
                4 & 0 & 0 & 0 & 0 & 1 \\
                5 & 1 & 1 & 1 & 0 & 0 \\
                6 & 0 & 0 & 0 & 1 & 1\bigstrut[b]\\
            \end{block}
        \end{blockarray}\vspace*{-1.25\baselineskip}
        \]
        
        & 
        \[
        \begin{blockarray}{ccccccc}
             & 1 & 2 & 3 & 4 & 5 & 6\\
            \begin{block}{c[cccccc]}
                1 & 0 & 1 & 0 & 1 & 0 & 1\bigstrut[t] \\
                2 & 1 & 0 & 0 & 1 & 0 & 0\\
                3 & 0 & 0 & 0 & 0 & 0 & 0\\
                4 & 1 & 1 & 0 & 0 & 1 & 0\\
                5 & 0 & 0 & 0 & 1 & 0 & 0\\
                6 & 1 & 0 & 0 & 0 & 0 & 0\bigstrut[b]\\
            \end{block}
        \end{blockarray}\vspace*{-1.25\baselineskip}
        \] \\
    \end{tabular}
\end{center}

\section{Algoritmo}
\subsection{Principi}
L’algoritmo risolvente proposto (in due versioni) sfrutta l’ordine degli insiemi entro la collezione N (detto \underline{ordine lessicografico}).\\
Esso produce incrementalmente la matrice di compatibilità e intanto analizza via via gli aggregati di insiemi della collezione N la compatibilità reciproca dei quali sia già stata appurata.
\begin{itemize}
    \item Nessun aggregato contiene come elemento un insieme vuoto
    \item Nessun aggregato contiene come elemento un insieme coincidente con M
\end{itemize}
L’operato dell’algoritmo può essere descritto in termini di \underline{\textbf{esplorazione di alberi}}, dove un nodo d’albero rappresenta un aggregato di uno o più insiemi della collezione.
\begin{itemize}
    \item Ciascun albero ha per radice un insieme distinto della collezione e contiene tutti e soli gli aggregati costituiti da tale insieme e da insiemi che lo precedono secondo l’ordine lessicografico degli insiemi.
    \item A ogni nodo d’albero è associato l’insieme degli identificatori degli insiemi che appartengono all’aggregato a cui il nodo si riferisce
    \item A ciascuna radice è associato un insieme (di identificatori) singoletto $\{i\}$, $1 \leq i \leq |N|$, dove $A[i]$ non è l’insieme vuoto né coincide con M: l’albero avente tale radice si dice «\underline{radicato in i}»
    \item Gli alberi vengono visitati per valori crescenti degli identificatori delle loro radici (ordine lessicografico)
\end{itemize}

\subsection{Visita degli Aggregati}
\paragraph{Due insiemi.}
Durante la visita dell’albero radicato in $i$ vengono inizializzate le caselle della matrice di compatibilità della colonna relativa a $i$ (prima la casella $B[1,i]$, poi $B[2,i]$ ecc. fino ad assegnare un valore alla casella $B[i-1,i]$): questo implica \textbf{la visita di tutti i nodi corrispondenti ad aggregati di cardinalità 2}
\paragraph{Tre o più insiemi.}
Ciascun nodo corrispondente a un aggregato di cardinalità $>2$ viene invece visitato \textbf{solo se gli insiemi che lo costituiscono sono compatibili a due a due}, secondo quanto riportato nella matrice B
\paragraph{Sotto-colonne}
Nello pseudocodice, con $B[1...i, j]$ si indica la sotto collezione di N costituita dagli insiemi compatibili con l’insieme $A[j]$ i cui indici vanno da 1 a $i (j>i)$

\subsection{Ordine di Visita e Analisi}
\begin{figure}[H]
  \centering
  \includesvg[inkscapelatex=false, scale=0.60]{figures/Tree.svg}
  \caption{Esempio Albero dell'Algoritmo}
\end{figure}
Vicino ai nodi è riportato l’ordine di visita dell’albero radicato \textbf{nell’insieme identificato da 5}, assumendo che gli insiemi non vuoti né coincidenti con M che lo precedono nella collezione N siano quelli aventi gli identificatori da 1 a 4 e che l’ordine entro la collezione rispecchi l’ordine dei valori interi.
\begin{itemize}
    \item L’ordine di visita illustrato garantisce che le caselle $B[1,5]$, $B[2,5]$, $B[3,5]$ e $B[4,5]$ vengano inizializzate nell’ordine indicato
    \item Dopo che sono state inizializzate le caselle $B[1,5]$ e $B[2,5]$ è possibile analizzare l’aggregato 521 perché le compatibilità degli aggregati 51 e 52 sono note, così come quella dell’aggregato 21, poiché l’albero radicato in 2 è stato visitato prima di quello radicato in 5
    \item Dopo che sono state inizializzate le caselle $B[1,5]$, $B[2,5]$ e $B[3,5]$ è possibile analizzare gli aggregati 531 e 532 perché le compatibilità degli aggregati 51, 52 e 53 sono note, così come quelle degli aggregati 31 e 32
\end{itemize}
\begin{figure}[H]
  \centering
  \includesvg[inkscapelatex=false, scale=0.60]{figures/Tree2.svg}
  \caption{Esempio Albero dell'Algoritmo con nodi scartati}
\end{figure}
\begin{itemize}
    \item Se la visita di un nodo del primo livello di un albero (cioè l’aggregato $ij$ , figlio della radice $i$) evidenzia che gli insiemi $A[i]$ e $A[j]$ hanno \textbf{un'intersezione non vuota} oppure che \textbf{la loro unione coincide con M}, i discendenti del nodo $ij$ non vengono esplorati così come non vengono esplorati gli aggregati che contengono sia $i$ sia $j$.
    \begin{itemize}
        \item Per Esempio se l’insieme 5 ha una intersezione non vuota con l’insieme 3 oppure se l’unione degli insiemi 5 e 3 coincide con M, i nodi discendenti del nodo 53 non vengono visitati così come non vengono visitati i nodi appartenenti all’intero sottoalbero radicato in 543, come evidenziato dal nuovo contorno di tali nodi
    \end{itemize}
    \item Se la visita di un nodo appartenente a un livello successivo al primo evidenzia che l’aggregato rappresentato da tale nodo \textbf{è una partizione}, i discendenti di tale nodo non vengono esplorati.
    \begin{itemize}
        \item Per Esempio se l’aggregato 542 è una partizione, i suoi nodi discendenti (in questo caso uno solo) non vengono visitati
    \end{itemize}
\end{itemize}

\newpage
\subsection{Algoritmo Base - Pseudocodice}
\IncMargin{1em}
\begin{algorithm}
    \DontPrintSemicolon
    \SetKwArray{A}{A}\SetKwData{i}{i}\SetKwArray{Rows}{rows}
    \SetKwData{SetM}{M}\SetKwData{Cov}{COV}\SetKwArray{B}{B}
    \SetKwData{j}{j}\SetKwData{U}{U}\SetKwData{I}{I}\SetKwData{Itemp}{Itemp}
    \SetKwData{Utemp}{Utemp}\SetKwData{Inter}{Inter}\SetKwData{VarK}{k}
    \SetKwData{InterTemp}{Intertemp}
    \SetKwFunction{Esplora}{Esplora}
    \BlankLine
    \SetKwProg{Fn}{procedure}{}{}
    \Fn{EC(\A)}{
        \For{\i$\leftarrow1$ \KwTo \Rows{\A}}{
            \lIf(\tcc*[f]{\textbf{break} termina iterazione $i$-esima}){\A{\i}$==\varnothing$}{
                \textbf{break}
            }
            \lIf(\tcc*[f]{\Cov variabile globale}){\A{\i}$==$\SetM}{
                Put $\{i\}$ in \Cov,
                \textbf{break}
            }
            In \B aggiungere la colonna relativa ad \i\;
            \For{\j$\leftarrow1$ \KwTo \i$-1$}{
                \eIf{\A{\j}$\cap$\A{\i}$\neq\varnothing$}{
                    \B{\j,\i}$\leftarrow0$
                }{
                    \I$\leftarrow\{$\i,\j$\}$, \U$\leftarrow$\A{\i}$\cup$\A{\j}\;
                    \eIf{\U$==$\SetM}{
                        inserire \I in \Cov, \B{\j,\i}$\leftarrow0$
                    }{
                        \B{\j,\i}$\leftarrow1$, \Inter $\leftarrow$ \B{$1...$\j$-1$, \i}$\cap$\B{$1...$\j$-1$, \j}\;
                        \lIf{\Inter$\neq\varnothing$}{
                            \Esplora{\I, \U, \Inter}
                        }
                    }
                }
            }
        }
    }
    \Fn{Esplora(\I, \U, \Inter)}{
        \ForAll(\tcc*[f]{Ordine lessicografico del valore di \VarK}){\VarK$\in $\Inter }{
            \Itemp$\leftarrow$\I$\cup\{$\VarK$\}$, 
            \Utemp$\leftarrow$\U$\cup$\A{\VarK}\;
            \eIf{\Utemp$==$\SetM}{
                inserire \Itemp in \Cov
            }{
                \InterTemp $\leftarrow$ \Inter $\cap$ \B{$1...$\VarK$-1$,\VarK}\;
                \lIf{\InterTemp$\neq\varnothing$}{\Esplora{\I, \U, \Inter}}
            }
        }
    }
    \caption{Algoritmo Base}\label{base_alg}
\end{algorithm}\DecMargin{1em}

\newpage
\subsection{Algoritmo Plus - Pseudocodice}
\IncMargin{1em}
\begin{algorithm}
    \DontPrintSemicolon
    \SetKwArray{A}{A}\SetKwArray{Rows}{rows}\SetKwArray{B}{B}
    \SetKwArray{Card}{card}
    \SetKwData{SetM}{M}\SetKwData{Cov}{COV}\SetKwData{i}{i}
    \SetKwData{j}{j}\SetKwData{CardU}{cardU}\SetKwData{I}{I}\SetKwData{Itemp}{Itemp}
    \SetKwData{Utemp}{Utemp}\SetKwData{Inter}{Inter}\SetKwData{VarK}{k}
    \SetKwData{CardTemp}{cardtemp}
    \SetKwData{InterTemp}{Intertemp}
    \SetKwFunction{Esplora}{Esplora$^+$}
    \BlankLine
    \SetKwProg{Fn}{procedure}{}{}
    \Fn{EC$^+$(\A)}{
        \For{\i$\leftarrow1$ \KwTo \Rows{\A}}{
            \lIf(\tcc*[f]{\textbf{break} termina iterazione $i$-esima}){\A{\i}$==\varnothing$}{
                \textbf{break}
            }
            \lIf(\tcc*[f]{\Cov variabile globale}){\A{\i}$==$\SetM}{
                Put $\{i\}$ in \Cov,
                \textbf{break}
            }
            \Card{\i}$\leftarrow|$\A{\i}$|$\;
            In \B aggiungere la colonna relativa ad \i\;
            \For{\j$\leftarrow1$ \KwTo \i$-1$}{
                \eIf{\A{\j}$\cap$\A{\i}$\neq\varnothing$}{
                    \B{\j,\i}$\leftarrow0$
                }{
                    \I$\leftarrow\{$\i,\j$\}$, 
                    \CardU$\leftarrow$\Card{\i}$+$\Card{\j}\;
                    \eIf{\CardU$==|$\SetM$|$}{
                        inserire \I in \Cov, \B{\j,\i}$\leftarrow0$
                    }{
                        \B{\j,\i}$\leftarrow1$, \Inter $\leftarrow$ \B{$1...$\j$-1$, \i}$\cap$\B{$1...$\j$-1$, \j}\;
                        \lIf{\Inter$\neq\varnothing$}{
                            \Esplora{\I, \CardU, \Inter}
                        }
                    }
                }
            }
        }
    }
    \Fn{Esplora$^+$(\I, \CardU, \Inter)}{
        \ForAll(\tcc*[f]{Ordine lessicografico del valore di \VarK}){\VarK$\in $\Inter }{
            \Itemp$\leftarrow$\I$\cup\{$\VarK$\}$, 
            \CardTemp$\leftarrow$\CardU$+|$\A{\VarK}$|$\;
            \eIf{\CardTemp$==|$\SetM$|$}{
                inserire \Itemp in \Cov
            }{
                \InterTemp $\leftarrow$ \Inter $\cap$ \B{$1...$\VarK$-1$,\VarK}\;
                \lIf{\InterTemp$\neq\varnothing$}{\Esplora{\I, \CardTemp, \Inter}}
            }
        }
    }
    \caption{Algoritmo Plus}\label{plus_alg}
\end{algorithm}\DecMargin{1em}