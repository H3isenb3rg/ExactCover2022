\chapter{Conclusioni e lavori futuri}
\section{Conclusioni e osservazioni}
I risultati ottenuti mostrano che l’algoritmo sia poco efficiente in termini di performance temporali al crescere della dimensione del problema: esistono infatti altri algoritmi atti a risolvere lo stesso problema molto più efficienti, come l’algoritmo di Knuth che adopera i Dancing Links.

Confrontando gli algoritmi base e plus quest’ultimo è risultato più veloce in tutti i test, andando inoltre ad ampliare il distacco dal base nel caso dell’implementazione della rappresentazione binaria dell’input.

Avendo scelto un linguaggio di programmazione ad alto livello come Python per la sua flessibilità e facilità di prototipazione e riuso del codice, nonchè per nostra maggiore familiarità con il linguaggio, è possibile che esso contribuisca alle non ottimali performance del programma.

\section{Lavori futuri}
Riteniamo che all’algoritmo in se non si possano effettuare variazioni che vadano ad influire particolarmente sulla velocità di esecuzione (fermo restando di utilizzare esattamente questo algoritmo senza variarne il funzionamento basilare).

Si potrebbero svolgere nuovi test sia con una versione più aggiornata di Python, in particolare la 3.11, ultima uscita, che ha aumentato le performance generali del linguaggio, oppure con un diverso linguaggio di programmazione a più basso livello come C.
